\chapter{Moments of the RT and GRT}{}

In this chapter we discuss some necessary conditions for the the determinacy of certain multivariate moment problems by way of the Radon and Gaussian Radon transforms.

\section{Projection moments and Petersen's theorem}

\begin{proposition}
  Let $c_\alpha (\omega) = \int_{-\infty}^\infty R_f(\omega, p) p^k dp$ be the projection moments of $f$ at a fixed $\omega$, and $c_\alpha$ the multivariate moments of $f$. Then
  \[
      c_k(\omega) = \sum_{|\alpha| = k}\binom{k}{\alpha} \omega^\alpha c_\alpha
  \]
  where $\binom{k}{\alpha} = \frac{k!}{\alpha_1! \alpha_2! \cdots \alpha_n!}$ are multinomial coefficients.
\end{proposition}

\begin{proof}
  By the slice theorem (\ref{eq:ST}) with $F(p) = p^k$,
  \[
    \int_{-\infty}^\infty R_f(\omega, p) p^k ~dp 
    = \int_{\RR^n} f(x) \langle x, \omega \rangle^k ~dx.
  \]
  Now $\langle x, \omega \rangle^k = {(x_1 \omega_1 + \cdots + x_n \omega_n)}^k$ has the multinomial expansion
  \[
    \langle x, \omega \rangle^k = \sum_{|\alpha| = k}\binom{k}{\alpha} x^\alpha\omega^\alpha.
  \]
  Thus after a bit of rearranging we get
  \begin{align*}
    \int_{\RR^n} f(x) \langle x, \omega \rangle^k ~dx
    &= \int_{\RR^n} f(x) \sum_{|\alpha| = k}\binom{k}{\alpha} x^\alpha \omega^\alpha ~dx \\
    &= \sum_{|\alpha| = k}\binom{k}{\alpha} \omega^\alpha \int_{\RR^n} f(x) x^\alpha ~dx,
  \end{align*}
  where the integrands are precisely the $k$th degree multivariate moments of $f$.
\end{proof}

Similarly, moments of the GRT (Gaussian projection moments) can be expressed in terms of multivariate gaussian moments.

\begin{proposition}
  Let $c_k^G(\omega) = \int_{-\infty}^\infty GR_f(\omega, p) p^k w(p) dp$ be the Gaussian moments of the GRT of $f$ at a fixed $\omega$. Let $c^G_\alpha = \int_{\RR^n} f(x) w_n(x) x^\alpha dx$ be the Gaussian multivariate moments of $f$. Then
  \[
    c^G_k(\omega) = \sum_{|\alpha| = k}\binom{k}{\alpha} \omega^\alpha c^G_\alpha.
  \]
\end{proposition}


\begin{proof}
  The proof follows as it did for the RT.\@ This time we apply the GRT slice theorem (\ref{eq:GST}) with $F(p) = p^k$, 
  \begin{align*}
    \int_{-\infty}^\infty GR_f(\omega, p) p^k w(p) ~dp
    &= \int_{\RR^n} f(x) \langle x, \omega \rangle^k w_n(x) ~dx
  \end{align*}
  Again we use the multinomial expansion of $\langle x, \omega\rangle^n$ and rearrange:
  \[
    \int_{\RR^n} f(x) \langle x, \omega \rangle^k w_n(x) ~dx
    = \sum_{|\alpha| = k} \binom{k}{\alpha} \omega^\alpha \int_{\RR^n} f(x) w_n(x)x^\alpha dx. 
  \]
  Thus
  \[
    c^G(\omega) = \sum_{|\alpha| = k} \binom{k}{\alpha} \omega^\alpha c^G_\alpha.
  \]
\end{proof}

\begin{myexample}
  Let $e_1, e_2, \ldots, e_n \in S^{n-1}$ be the the standard basis for $\RR^n$,
  \[
    (1, 0, \ldots, 0),~ (0, 1, \ldots, 0),~ \ldots,~ (0,0, \ldots, 1)
  \]
  Then $\langle x, e_i \rangle = x_i$ is the natural projection of $\RR^n$ onto the $e_i$ axis. The standard projection moments can be calculated as follows
  \begin{align*}
    c_k(e_i) 
    &= \sum_{|\alpha| = k} \binom{k}{\alpha} e_i^\alpha c_\alpha \\
    &= c_{ke_i}
  \end{align*}
  since $e_i^\alpha = 0$ unless $\alpha = ke_i$.
\end{myexample}

The following theorem, due to Petersen \cn, gives a way us to reduce the question of determinacy for multivariate moment problems to the classical case.

\begin{proposition}[Petersen's theorem]
  Let $\mu$ be a Borel measure with finite moments on $\RR^n$, and $e_1, \ldots, e_n$ the standard basis for $\RR^n$. If each $R_\mu^{e_1}, \ldots, R_\mu^{e_n}$ is determinate, then $\mu$ is determinate.
\end{proposition}

\begin{proof}
  An outline of the proof is as follows. Preliminarily, note that the solution set $[\mu]$ of Borel measures with equivalent moments to $\mu$ is convex \cn, and a $\mu$ is an extreme point in $[\mu]$ if and only if polynomials are dense in $L^1(\RR^n, \mu)$ \cn. Thus it suffices to show that polynomials are dense in $L^1(\RR^n, \mu')$ for any $\mu' \in [\mu]$.
  
  The family of products of continuous functions of compact support $f(x) = \prod_{i=1}^n f_i(x_i)$ where each $f_i \in C_c(\RR)$, is dense in $L_1(\mu)$ \cn. Furthermore, since $R_\mu^{e_i}$, $i = 1, \ldots, n$ are determinate, polynomials are dense in each $L^2(\RR, R_\mu^{e_i})$ \cn. Petersen constructs a series of polynomials $P_i : \RR \rightarrow \RR$ such that the polynomial product $P(x) = \prod_{i = 1}^n P_i(x_i)$ arbitrarily close to $f$ in $L^1(\mu)$. Thus polynomials are dense in $L^1(\RR^n, \mu)$ and the moment problem is determinate. \pn 
  
  Note Schm\"udgen proves this via Borel characteristic functions. Not clear what the benefit is.
\end{proof}

\begin{remark}
  Conjecture? We expect this result to hold true for any orthonormal basis, and likely any basis. Will have to check this. Arguable benefit for us is that we in theory can use any $n$ linearly independent projections to reconstruct $\mu$.
\end{remark}

\begin{corollary}
  If $\mu$ is compactly supported Borel measure on $\RR^n$, then the multivariate moment problem is determinate.
\end{corollary}

\begin{proof}
  It suffices to note that the projections $R_\mu^{e_i}$ are compactly supported Borel measures on $\RR$, and thus determinate by \cn.
\end{proof}

\section{Determinate Gaussian projection moments}

We show in section (1.2) that any function $f$ in the weighted space $L^2(\RR, \gamma)$ is determinate. By Petersen's theorem a function who's Gaussian projections lie in the same space is likewise determinate. Of interest to us are two families of determinate functions on $\RR^n$: First, bounded functions, including characteristic functions of Borel sets. And second, locally bounded functions with at most polynomial growth. 

\begin{theorem}
  If $f : \RR^n \to [0, \infty)$ is a bounded measurable function then its Gaussian moments are determinate as well as its Gaussian projection moments.
\end{theorem}

\begin{proof}
  The characteristic function $f$ of $A$ is measureable and certainly has at most polynomial growth. In fact, this result is easily proven without theorem 2.2.2.\ in one line:
  \[
    GR_f(\omega, p) = \int\mclimits_{\inner{x, \omega} = p} f(x) w_{n-1}(x - p\omega) dx \leq \int\mclimits_{\inner{x, \omega} = p} w_{n-1}(x - p\omega) = 1
  \]
\end{proof}

\begin{lemma}
  If a Borel measurable $f : \RR^n \to [0, \infty)$ is locally bounded and has at most polynomial growth then for all $\omega \in S^{n-1}$, the GRT $GR_f(\omega, p): \RR \to [0, \infty)$ converges, is locally bounded and has at most polynomial growth as $p \rightarrow \pm \infty$. 
\end{lemma}

\begin{proof}
  Suppose that for some $D > 1$ and $m \in \NN$ we have
  \[
    f(x) \leq \norm{x}^m, \qquad \text{for } \norm{x} \geq D.
  \]
  Then for $|p| \geq D$, 
  \begin{align*}
    GR_f(\omega, p)
      &= \int\mclimits_{\inner{x, \omega} = p} f(x) w_{n-1}(x - p\omega) ~dx \\
      &\leq \int\mclimits_{\inner{x, \omega} = p} \norm{x}^m w_{n-1}(x - p\omega) ~dx \\
      &= \int\mclimits_{\inner{x, \omega} = 0} \norm{x + p\omega}^m w_{n-1}(x) ~dx \\
      &= \sum_{k = 0}^m \binom{m}{k} |p|^{m - k} \int\mclimits_{\inner{x, \omega} = 0} \norm{x}^k w_{n-1}(x) ~dx \\
  \end{align*}
  where the integrals are finite, and thus $GR_f(\omega, p)$ is dominated by at most some constant multiple of $|p|^m$. This implies local boundedness for $|p| \geq D$. For $|p| < D$ we note that
  \begin{align*}
    \int\mclimits_{\inner{x, \omega} = p} f(x) w_{n-1}(x - p\omega) ~dx
      &= \int\mclimits_{\substack{\inner{x, \omega} = p \\ |x| < D}} f(x) w_{n-1}(x - p\omega) ~dx + \int\mclimits_{\substack{\inner{x, \omega} = p \\ |x| \geq D}} f(x) w_{n-1}(x - p\omega) ~dx \\
      &\leq \int\mclimits_{\substack{\inner{x, \omega} = p \\ |x| < D}} f(x) w_{n-1}(x - p\omega) ~dx + \int\mclimits_{\substack{\inner{x, \omega} = p \\ |x| \geq D}} \norm{x}^m w_{n-1}(x - p\omega) ~dx
  \end{align*}
  where the first integral is bounded by the locally boundedness of $f$ and the second converges.
\end{proof}

\begin{theorem}
  If a Borel measurable $f : \RR^n \to [0, \infty)$ is locally bounded and has at most polynomial growth then its Gaussian moments are determinate. Furthermore for any $\omega \in S^{n-1}$ the GRT $GR_f(\omega, p)$ has determinate Gaussian moments.
\end{theorem}

\begin{proof}
  By the lemma and proposition \cn it suffices to justify that a locally bounded measurable function $f(p) : \RR \to [0, \infty)$ with at most polynomial growth is in the weighted $L^2$ space $L^2(\RR, w)$. Suppose $|f(p)| < |x|^m$ for all $|x| > D$ and some $m \in \NN_0$. Then
  \begin{align*}
    \int_{-\infty}^\infty |f(p)|^2 w(p) ~dp 
      &= \int_{|x| \leq D} |f(p)|^2 w(p) ~dp + \int_{|x| > D} |f(p)|^2 w(p) ~dp 
  \end{align*} 
  The first integral converges by local boundedness and the second by polynomial growth.
\end{proof}

% \begin{corollary}
%   If $f \in L^2(\RR^n, \gamma^n)$, then $f$ is determinate.
% \end{corollary}

% \begin{proof}
%   % This follows from the product decomposition of the the Gaussian density
%   % \[
%   %   w_n(x) = \prod_{i = 1}^n w(x_i)
%   % \]
%   % so that
%   % \[
%   %   R_f(e_i, p) = \int_{x_i=p} f(x) w_n(x) dx
%   % \]
%   \pn
% \end{proof}

% \begin{proposition}
%   A function $f(p) \in L^2(\mathbb{R}, e^{-p^2}dp)$ such that the moments 
%   \begin{equation}
%     \label{eq:3}
%     c_k = \int_{-\infty}^\infty p^k f(p) w(p) ~dp, ~ k = 0, 1 \ldots
%   \end{equation}
%   are finite, is uniquely determined by those moments.
% \end{proposition}
    
% \begin{proof}
%   It is sufficient to show that if $c_k = 0$ for all $k \geq 0$, then $f \equiv 0$ a.e. The Hermite moments are just linear combinations of $0$,
%   \[
%       \int_{-\infty}^\infty H_k(p) f(p) w(p) ~dp = 0, ~ k \geq 0
%   \]
%   where $H_k(p)$ is the Hermite polynomial of order $k$. Since the Hermite polynomials are complete in the space $L^2(\mathbb{R}, e^{-p^2}dp)$ then $f \equiv 0$.
% \end{proof}
