%%
%% BOF settings.tex
%%


%%
%% MARGIN DECLARATIONS
%%
%% The Graduate School requires the following minimal margins: 1in on top, right and bottom 
%% and 1.5in on the left. And yes, this feature, they check.
          
\voffset -15mm
\hoffset 2mm
\setlength{\textheight}{220mm}
\setlength{\textwidth}{145mm}


%%
%% LaTeX EXTENSIONS
%%

% Most of these packages are not needed for the UConn Thesis class, but are quite useful
% for writing your thesis. Documentation about these packages can be found in your
% average LaTeX manual or by doing the following Google search :
% "latex <name-of-the-package> documentation site:www.tug.org" 

\usepackage{amsmath} % AMS math commands like \frac{}{} 
\usepackage{amscd} % commutative diagrams made easy -  has to be loaded after amsmath
\usepackage{amsthm}  % theorem, lemma, definition, ... -  has to be loaded after amsmath
\usepackage{calc} % minimal calculator operations

\usepackage{graphicx} % Imports figures : pdf, jpg, png, tif (one f)
                      % Use the package graphics if you use eps files.
                      % Warning : It is very difficult to use both graphics and graphicx.

\usepackage{enumerate} % extends enumerate environment
\usepackage{dsfont, marvosym, latexsym, amssymb, wasysym, mathbbol} % extra symbols 
\usepackage{color} % use colors red/blue/green/yellow/magenta
\usepackage{fancyhdr} % headers
\usepackage{fancybox} % oval boxes and stuff
\usepackage{ifthen} % if-then-else structure
\usepackage{rotate} % rotate and mirror commands

% Caligraphic Alphabet - used by the definitions of \cA, \cB, \cC and so on.
\usepackage[mathcal]{euscript}       
% Very Curly Caligraphic Alphabet - used by the definitions of \ccA, \ccB, \ccC and so on.
\usepackage{mathrsfs}       

%%
%% FONT DECLARATIONS
%%

% There are many fonts out there for LaTeX. If you want to see what your thesis looks like
% in this font, uncomment one of them. If all of them remain inactive, the system default will
% be chosen (usually Courier). A few warnings:
% (a) The Graduate School wants the fonts to be simular to Times New Roman, Helvetica or Arial.
%     Yes, Courier is fine. As you will notice many of the options below are only slightly different.
%     Of these fonts, cmbright and pxfonts are slightly outlandish and bookman is somewhat (too?) heavy.
% (b) Changeing from one font to another can give (rather small) changes in layout.
% (c) Some of these declarations need LaTeX extensions, so don't move this section before the previous one. 
%
% Choose one of the following:
% avant, bookman, charter, cmbright, courier, helvet, newcent, times, palatino, pxfonts

\usepackage{times}

%%
%% JASPER'S PACKAGES
%%
\usepackage{amsrefs} % ams style references
\usepackage{ulem} % strikeout and various underline types
\usepackage{tikz} % tikz figures
\usepackage{mathtools} % mathclap primarily

%%
%% DEF. THM. LEMMA AND THE REST 
%%

% Comment the following line out if you want "Theorem 1.2.3" and not "1.2.3 Theorem"
% \swapnumbers 

% Equations will be numbered with their section number, i.e. (1.1.1), (1.1.2), (1.2.1) and so on.
% Commenting this out put will result in all equations being numbered (1), (2), (3), ... (999) which
% is not the best idea for a document as long as a thesis.
\numberwithin{equation}{section}                      

% There are three theoremstyles: plain, definition and remark.
% Notice that "definition" produces the most standard output.
\theoremstyle{definition}

% If you want to use subsections, change the word "section" to "subsection" in the following declaration.
% however, be advised that then all numbering labels will contain 4 numbers and that you should divide
% up _every_ section into subsections or get labels like (2.2.0.1), (2.2.0.2), (2.2.0.3) etc.
\newtheorem{theorem}{Theorem}[section]

% These commands use the amsthm package. If you want to add a declatarion, it has to be of the form
% \newtheorem{environment-name}[theorem]{Word-you-want-printed}
\newtheorem{definition}[theorem]{Definition}
\newtheorem{definitions}[theorem]{Definitions}
\newtheorem{lemma}[theorem]{Lemma}
\newtheorem{myexample}[theorem]{Example}
\newtheorem{myexamples}[theorem]{Examples}
\newtheorem{proposition}[theorem]{Proposition}
\newtheorem{property}[theorem]{Property}
\newtheorem{properties}[theorem]{Properties}
\newtheorem{corollary}[theorem]{Corollary}
\newtheorem{notation}[theorem]{Notation}
\newtheorem{notations}[theorem]{Notations}
\newtheorem{remark}[theorem]{Remark}
\newtheorem{remarks}[theorem]{Remarks}
\newtheorem{introduction}[theorem]{Introduction}
\newtheorem{framework}[theorem]{Framework}
\newtheorem{conditions}[theorem]{Conditions}

% The Q.E.D. symbol can be defined in any way.  
% Commenting this out will give the default which is an empty square.
% \renewcommand{\qedsymbol}{{\slshape Qed}}


%%
%% PARINDENT 0
%%

% Needed to keep titles of Section and Subsections in their place,
% amongst other reasons. Do not change.
\parindent 0pt 


%%
%% DEFAULT ENUMERATE FORMATS
%%

\renewcommand\theenumi{\roman{enumi}}
\newcommand\labelenumi{(\theenumi)\ \ }
\renewcommand\theenumii{\alph{enumii}}
\newcommand\labelenumii{(\theenumii)\ }
\renewcommand\theenumiii{\roman{enumiii}}
\newcommand\labelenumiii{(\theenumiii)\ }
\renewcommand\theenumiv{\Alph{enumiv}}
\newcommand\labelenumiv{(\theenumiv)\ }
\leftmargini 30pt 
\leftmarginii 15pt 
\leftmarginiii 15pt 
\leftmarginiv 10pt 
\labelsep 5pt


%%
%% DEFAULT ITEMIZE FORMATS
%%

\newcommand\labelitemi{\textbullet}
\newcommand\labelitemii{\normalfont\bfseries\textendash}
\newcommand\labelitemiii{\textasteriskcentered}
\newcommand\labelitemiv{\textperiodcentered}


%%
%% EOF settings.tex
%%


%%
%% JASPER'S COMMANDS
%%

\newcommand{\RR}{\mathbb{R}} % Real numbers
\newcommand{\CC}{\mathbb{C}} % Complex numbers
\newcommand{\NN}{\mathbb{N}} % Natural numbers
\newcommand{\PP}{\mathbb{P}} % Projective Space
\newcommand{\re}{\text{Re}~} % Real part
\newcommand{\im}{\text{Im}~} % Imaginary part

\def\mclimits_#1{\limits_{\mathclap{#1}}} % Collapsable limits