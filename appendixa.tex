% \chapter{Long Title}{Short Title}
% The Long Title will appear on the first page of the chapter.
% The Short Title will appear in the table of contents.
% If the Long Title isn't all that long, you can just call
% \chapter{Long Title}{} and the same title will appear in
% both places.


\appendix{Tensor notation}{}

Let $V$ be a finite dimensional real vector space with basis $\{e_1, e_2, \ldots, e_n\}$. The tensor power $V^{\otimes k}$ is a real vector space, with basis
\[
  \{e_{i_1} \otimes e_{i_2} \otimes \cdots \otimes e_{i_k}\}_{(i_1, \ldots i_k) \in \{1, \ldots n\}^k}
\]
By convention we order this basis lexicographically. For example in the case $n = 3$, $k = 2$, 
\begin{align*}
  \{ e_1 \otimes e_1,~
  e_1 \otimes e_2,~
  e_1 \otimes e_3, \\
  e_2 \otimes e_1,~
  e_2 \otimes e_2,~
  e_2 \otimes e_3, \\
  e_3 \otimes e_1,~
  e_3 \otimes e_2,~
  e_3 \otimes e_3, \}
\end{align*}
For convenience we will use the vector notation ${\bf e} = (e_1, \ldots, e_n)$ for the basis of $V$ and 
\[
  \bf{e}_{\otimes i} = e_{i_1} \otimes e_{i_2} \otimes \cdots \otimes e_{i_k}
\]
for the elements of the induced basis of $V^{\otimes k}$, where $i = (i_1, i_2, \ldots i_k) \in \{1, 2, \ldots, n\}^k$.
There is a natural embedding $V^k \hookrightarrow V^{\otimes k}$ given by
\[
  (v_1, v_2, \ldots, v_k) \mapsto v_1 \otimes v_2 \otimes \cdots \otimes v_k
\]
and the tensor power satisfies the following universal property: Any ``multilinear'' map out of $V^k$ can be factored as a linear map out of $V^{\otimes k}$ composed with the inclusion above. 

\needed{Commutative diagram}

% Take for a simple example the multiplication map $m: \RR^2 \to \RR$,
% \[
%   (v_1, v_2, \ldots, v_k) \mapsto v_1 v_2 \cdots v_k.
% \]


Given a linear transformation $T : V \to V$, the tensor power $T^{\otimes k}$ is the unique linear map $V^{\otimes k} \to V^{\otimes k}$ such that
\[
  T^{\otimes k} (\bf{e}_{\otimes i}) 
  = (Te_{i_1}) \otimes \cdots \otimes (Te_{i_k})
\]
In the basis $\{\bf{e}_{\otimes i}\}$ ordered lexicographically the matrix representation for $T^{\otimes k}$ is given explicitly by the $k$-fold Kronecker product of the matrix for $T$ in the basis $\bf{e}$.
Tensor powers of linear maps commute with transposes
\[
  (T^{\otimes k})^\top = (T^\top)^{\otimes k}
\]
If $V$ has an inner product $\inner{\cdot, \cdot}_V$ then an inner product on $V^{\otimes k}$ can be defined by
\[
  \inner{\bf{e}_i, \bf{e}_{i'}}_{V^{\otimes k}}
  = \inner{e_{i_1}, e_{i'_1}}_V \cdots \inner{e_{i_k}, e_{i'_k}}_V.
\]
We will generally omit the subscripts and write $\inner{\cdot,\cdot}$ for either inner product, determined from context. For any linear transformation $T : V \to V$ we have
\[
  \inner{T^{\otimes k} (\bf{e}_i), \bf{e}_{i'}} = \inner{\bf{e}_i, (T^{\top})^{\otimes k}(\bf{e}_{i'})}.
\]
In particular for an orthogonal transformation $R \in O(n)$ on $\RR^n$ we have
\[
  \inner{R^{\otimes k} (\bf{e}_i), \bf{e}_{i'}} = \inner{\bf{e}_i, (R^{-1})^{\otimes k}(\bf{e}_{i'})}
\]
and for a projection $P$,
\[
  \inner{P^{\otimes k} \bf{e}_i, \bf{e}_{i'}} = \inner{\bf{e}_i, P^{\otimes k}\bf{e}_{i'}}.
\]
Note that $\inner{T^{\otimes k}(\bf{e}_i), \bf{e}_{i'}} = \inner{Te_{i_1}, e_{i'_1}} \cdots \inner{Te_{i_k}, e_{i'_k}}$ is a product of entries in the matrix representation for $T$ with basis $\bf{e}$.

The inner product allows for a convenient expression of polynomials. Henceforth we take $V = \RR^{n}$. For $x \in \RR^{n}$ a multi index $\alpha \in \NN_0^n$ of degree $|\alpha| = k$ we have
\[
  x^\alpha 
  = x_1^{\alpha_1} x_2^{\alpha_2} \cdots x_n^{\alpha_n}
  = \prod_{i = 1}^{n} \inner{x, e_i}^{\alpha_n}
  = \inner{x^{\otimes k}, \bf{e}^{\otimes \alpha}}
\]
where we write
\[
  x^{\otimes k} = \underbrace{x \otimes x \otimes \cdots \otimes x}_{\hbox{$k$ times}}
\]
and 
\begin{align*}
  {\bf e}^{\otimes \alpha}
  &= e_1^{\otimes \alpha_1} \otimes \cdots \otimes e_n^{\otimes \alpha_n}
  \\
  &= \underbrace{e_1 \otimes \cdots \otimes e_n}_{\hbox{$\alpha_1$ times}} \otimes \cdots \otimes \underbrace{e_n \otimes \cdots \otimes e_n}_{\hbox{$\alpha_n$ times}}
\end{align*}
For example, in $\RR^3$ the monomial $x^{(1, 1, 2)} = x_1x_2x_3^2$ can be written
\[
  x_1x_2x_3^2 = \inner{x \otimes x \otimes x \otimes x, e_1 \otimes e_2 \otimes e_3 \otimes e_3}
\]
Now the monomial representation is not unique. Indeed, since the tensor inner product is given as a product of euclidean products, we may as well imagine $\bf{e}^{\otimes \alpha}$ to be commutative. More precisely, for any permutation $\sigma \in S_k$ we have equivalently,
\[
  \inner{x^{\otimes k}, \bf{e}_{\otimes i}} = \inner{x^{\otimes k}, e_{\sigma^{-1}i_1} \otimes \cdots \otimes e_{\sigma^{-1}i_k}}
\]
for any $i \in \{1, \ldots, n\}^k$. The basis $\{{\bf{e}_{\otimes i}}\}_{i \in \{1, \ldots, n\}^k}$ can partitioned by equivalence under permutation to $\bf{e}^{\otimes \alpha}$ for each degree $k$ multiindex $\alpha$. Let
\[
  I(\alpha) = \{i \in \{1, \ldots, n\}^k : \inner{x^{\otimes k}, \bf{e}_{\otimes i}} = \inner{x^{\otimes k}, \bf{e}^{\otimes \alpha}}\}
\]
One can check that $\{I(\alpha)\}_{|\alpha| = k}$ is a partition of the $k$-tuples $\{1, \ldots, n\}^k$. For example with $n = 2, k = 2$, we have
\begin{align*}
  x_1^2 &= \inner{x \otimes x, e_1 \otimes e_1} \\
  x_1x_2 &= \inner{x \otimes x, e_1 \otimes e_2} = \inner{x \otimes x, e_2 \otimes e_1} \\
  x_2^2 &= \inner{x \otimes x, e_2 \otimes e_2}
\end{align*}
Thus it seems more appropriate to write monomials in terms of ``commutative'' symmetric tensors. We define the symmetrization of a basis tensor $\bf{e}_{\otimes i}$ by averaging over permutations
\[
  P_s \bf{e}_{\otimes i} := \frac1{k!} \sum_{\sigma \in S_k} e_{\sigma i_1} \otimes \cdots \otimes e_{\sigma i_k}.
\]
which can be seen as an orthogonal projection onto the subspace of symmetric tensors (tensors invariant under permutation).

The subspace of $k$-tensors invariant under permutation of indices is denoted $V^{\odot k} \subseteq V^{\otimes k}$. Elements may be defined by the projection,
\[
  v_1 \odot v_2 \odot \cdots \odot v_k = \frac1{k!} \sum_{\sigma \in S_k} v_{\sigma 1} \otimes v_{\sigma 2} \otimes \cdots \otimes v_{\sigma k}.
\]
The non-decreasing tuples $i = (i_1, \ldots i_k) \in \{1, \ldots, n\}^k$ indexes a standard basis, 
\[
  \bf{e}_{\odot i} 
  := e_{i_1} \odot \cdots \odot e_{i_k}, 
  \qquad 1 \leq i_1 \leq \cdots \leq i_k \leq n.
\]
Even better, there is a one to one correspondence between non-decreasing $k$-tuples and multi-indeces of degree $k$,
\[
  \tilde{\alpha} = (\underbrace{1, \ldots, 1}_{\alpha_1 ~\text{times}}, \underbrace{2, \ldots, 2}_{\alpha_2 ~\text{times}}, \ldots , \underbrace{n, \ldots, n}_{\alpha_n ~\text{times}})
\] 
so we may write the basis
\[
  \bf{e}^{\odot \alpha} 
  := e_{1}^{\odot \alpha_1} \odot \cdots \odot e_{n}^{\odot \alpha_n},
  \qquad \alpha \in \NN_0^n, |\alpha| = k
\]
The $k$-tensor inner product restricted to $V^{\odot k}$ says
\begin{align*}
  \inner{v_1 \odot \cdots \odot v_k, w_1 \odot \cdots \odot w_k} 
  &= \sum_{\sigma_1, \sigma_2 \in S_k} \prod \inner{v_{\sigma_1j}, w_{\sigma_2j}} \\
  &= \sum_{\sigma \in S_k} \prod \inner{v_{\sigma j}, w_j} \\
  &= \sum_{\sigma \in S_k} \prod \inner{v_, w_{\sigma j}}.
\end{align*}
Now the monomial expression in terms of symmetric tensors takes the simple form
\[
  x^\alpha = \inner{x^{\odot k}, \bf{e}^{\odot \alpha}}.
\]
Taken further, the correspondence between homogeneous polynomials of degree $k$ and symmetric $k$-tensors can be shown to be a vector space isomorphism, and extended to an isomorphism between the entire polynomial space and the graded space formed by symmetric tensors of all orders.

In the end one might wonder, if symmetric tensors are just polynomials in disguise, why bother with them?

% Let $V$ be vector space over a field $K$. A function $f$ out of $V^k$ is multilinear if it is linear in each entry, with all others held constant.
% \[
%   f(v_1, \ldots, av_i + bv'_i, \ldots, v_k) 
%   = af(v_1, \ldots, v_i, \ldots, v_k) + bf(v_1, \ldots, v'_i, \ldots, v_k)
% \]
% It is symmetric if it is unchanged by permutation of entries. 
% \[
%   f(v_1, \ldots, v_k) = f(v_{\sigma1}, \ldots, v_{\sigma k})
% \]
% The symmetric tensor power $V^{\otimes k}$ is a vector space defined by the universal property: Every symmetric multilinear map out of $V^k$ can be written uniquely 

