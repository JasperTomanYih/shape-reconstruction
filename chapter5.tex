\chapter{Krawtchouk Polynomials and a Discrete GRT}{}

Here we discuss ``discrete Radon transforms'' and propose a sort of ``discrete Gaussian Radon transform'' defined via this results of the Chapter 4.

\section{Discrete RTs of Diaconis and Beylkin}

\section{Krawtchouk polynomials}

The formula (\ref{eq:GRH}) gives us a path toward defining a discrete analogue to the Gaussian Radon transform. We will begin introducing the centered binomial density and corresponding Krawtchouk polynomials by way of a classic model in probability.

Consider flipping a loaded coin with probability $p \in [0,1]$ of landing on heads, and thus probability $(1 - p)$ of landing on tails. The probability of $k$ heads in $N$ coin flips is given by the binomial probability density,
\[
  \binom{N}{k}p^k(1-p)^{N-k}
\]
where $N \in \NN_0$ and $k = 0, 1, \ldots, N$. It is well known that the binomial distribution can be seen as a discrete analog to the Gaussian distribution. Thus we may expect that a discrete analog to the GRT may be defined via the binomial distribution.

In order to draw a clearer line between the binomial and Gaussian densities we make use of an alternate probabilistic model: the Bernoulli random walk. Picture a particle ``walking'' along the integer lattice $\ZZ$, beginning at the origin $x = 0$. At every step, we flip the loaded coin described above, stepping to the right if it lands on heads, and the left if tails. After $N$ flips, the particle's position is 
\[
  x = (\text{\# of heads}) - (\text{\# of tails}) = k - (N - k) = 2k - N
\]
Now the set of possible positions is $x \in [[N]] := \{-N, -N+2, \ldots, N\}$, and the probability of each is precisely the probability of landing $k = \frac{N + x}2$ heads, that is,
\[
  \binom{N}{\frac{N + x}2}p^{\frac{N + x}2}(1 - p)^{N - \frac{N + x}2}
  = \binom{N}{\frac{N + x}2}p^{\frac{N + x}2}(1 - p)^{\frac{N - x}2}
\]
We will define \emph{centered binomial density} supported on $[[N]]$ by
\[
  B_N(x) = \binom{N}{\frac{N+x}2}p^{\frac{N+x}2}(1-p)^{\frac{N-x}2}.
\]
Since the expected displacement at each (independent) step is $p - (1 - p) = 2p - 1$, we calculate the expected position after $N$ steps to be 
\[
  E(B_N) = 2Np-N.
\]
Furthermore the variance of each displacement step is $4p(1-p)$ so that the variance of $B_N$ is
\[
  Var(B_N) = 4Np(1-p)
\]
The earlier claim that the binomial density is a discrete analog to the Gaussian density can be made explicit by the following limit relation. Let \[
  z = \frac{x - E(B_N)}{\sqrt{Var(B_N)}} 
  = \frac{x - 2Np + N}{2\sqrt{Np(1-p)}}
\]
so that $z$ has mean $0$ and variance $1$ with respect to $B_n$. Equivalently we may set
\[
  x = z\sqrt{Var(B_N)} + E(B_N) = 2z\sqrt{Np(1-p)} + 2Np + N
\]
Then we have
\[
  \lim_{N \rightarrow \infty} B_N(x) = \frac{1}{(\sqrt{2\pi})}e^{-\frac{z^2}2} = w(z).
\]
This limit is perhaps one of the simplest cases of the robust central limit theorem, but can be proven in a number of more elementary ways \cn.

\begin{remark}
  Under an analogous limit a random walk process $x$ is transformed into a Brownian motion process $z$. \cn
\end{remark}

Having justified the the centered binomial densities $B_N(x)$ as discrete analogs of the Gaussian $w(z)$, we would like to define a sequence of polynomials orthogonal with respect to $B_N(x)$, as the Hermite polynomials are to $w(z)$. For convenience we start with a generating function 

\begin{definition}
  The Krawtchouk polynomials $K_k^{N,p}(x)$ may be defined by the generating function
  \[
    \sum_{k=0}^N K^{N,p}_k(x)y^k = (1 + y)^{\frac{N + x}2}(1 - y)^{\frac{N - x}2}
  \]
\end{definition}
% \begin{remark}
%   Note that this is an ordinary generating function, while that of the Hermite polynomials was a exponential generating function. Is this distinction consequential for us?
% \end{remark}

The polynomials $K^{N,p}_k(x)$ satisfy the orthogonality property
\[
  \sum_{x \in [[N]]} K^{N,p}_k(x) K^{N,p}_\ell(x) B_N(x) = 0, \qquad k \neq \ell
\]
and can be written explicitly as
\[
  K^{N,p}_k(x) = \binom{N}{k} \pFq{2}{1}{-k,(x - N)/2}{-N}{2}
\]

The Krawtchouk and Hermite polynomials are related by the same limiting process.
\begin{proposition}
  With $x = 2 z \sqrt{Np(1-p)} + 2Np + N$ we have the following limit relation:
  \[
    \lim_{N \rightarrow \infty} K^{N,p}_k(x) = H_k(z)
  \]
\end{proposition}

\begin{proof}
  \pn
\end{proof}

Now heuristically we expect a formula analagous to (\ref{eq:GRH}) in the from
\[
  GR_{K^{N,p}_\alpha}(\omega, p) = K^{N,p}_k(p)\omega^\alpha, \qquad |\alpha| = k
\]
where 
\[
  K^{N,p}_\alpha(x) = \prod_{i = 1}^n K^{N,p}_{\alpha_i}(x_i)
\]
We should resolve the conflicting variables both named $p$, perhaps the Krawtchouk $p$ ought to be a $q$ or isn't necessary in the first place.

The desired formula would represent a discrete analog for the Gaussian Radon transform on the finite cubic lattice $[[N]]^n$, however it is not yet clear how to interpret this. An immediate concern is that the cubic lattice does not admit ``hyperplanes'' for arbitrary $p$ and $\omega$. Indeed in the worst case, if $\omega$ is not rational (that is the ratios between coordinates of $\omega$ are not rational, or equivalently the hyperspherical coordinates of $\omega$ include an angle which is an irrational multiple of $\pi$) then a hyperplane orthogonal can contain at most one lattice point. In the case when $\omega$ is rational — in the sense above — we may restrict $p$ to the projection of the lattice onto the span of $\omega$, and sum over the lattice hyperplane $\inner{x, \omega} = p$ with respect to the binomial weight centered on $p\omega$. It remains to be seen whether such an interpretation agrees with the heuristic formula above.

Take the symmetric random walk, where the coin is fair ($p = 1 - p = \frac12$). The position after $N$ flips has distribution
\[
  B_N(x) = \frac1{2^N}\binom{N}{\frac{N+x}2}, \qquad x \in [[N]]
\]
with expected value $E(B_N) = 0$ and variance $Var(B_N) = N$. The limiting process is then given by $B_N(x) \rightarrow w(z)$ where 
\[
  z = \frac{x}{\sqrt{N}}
\]
and further $K^{N,\frac12}_k(x) \rightarrow H_k(z)$.

Now consider the following scenario in two dimensions. A particle starting at the origin $x = (x_1, x_2) = 0$ flips two fair coins, stepping down/up depending on the first coin, left/right depending on the second. This means the particle moves one unit up and right, up and left, down and right, or down and left, with equal probability. 

\section{The proposed discrete GRT}