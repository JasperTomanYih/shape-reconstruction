% \introchapter{Long Title}{Short Title}
% The Long Title will appear on the first page of the chapter.
% The Short Title will appear in the table of contents.
% If the Long Title isn't all that long, you can just call
% \introchapter{Long Title}{} and the same title will appear in
% both places.


\introchapter{\needed{TITLE?}}{}


\section{TO DO}
\begin{itemize}
    \item To write:
    \begin{itemize}
        \item Document. Fix bibliography
        \item \sout{Document. Fix section spacing and indentation. Ideally sections are left aligned with no indent, and the first paragraph of a section is not indented. Also should probably not indent after displaymath or similar mid-paragraph clauses (e.g.\ enumerate, itemize, etc)}
        \item Intro. More formulas in intro.
        \item Moments. Prove: Determinacy result for gaussian weighted L2.
        \item Moments. Find good reference for terminology and history. 
        \item \sout{Moments. Define problems for measures before functions.}
        \item GF/M.\@ Define Hermite polynomials, uni- and multi-variate.
        \item RT/GRT.\@ Slice theorem: connect sufficient conditions to Fubini conditions.
        \item RT/GRT.\@ Work on notation for Gaussian measure. In particular $w$ and $\omega$ are too similar. 
        \item RT/GRT.\@ Search for references on G slice theorem and G projection moments etc
        \item RT/GRT.\@ Expand background, historical context, etc\ldots
        \item Moments. Prove: Bounded regions are uniquely determined by moments? Maybe fits in RT/GRT since it can be formulated in terms of RT.\@
        \item Implementation. Output some figures from Mathematica. For example sample points, moment tables, and reconstructions.
        \item Example. Try to prove the conjecture (RT Hermite)
        \item Split Ch2 in half: Theory and application.
    \end{itemize}

    \item To research:
    \begin{itemize}
        \item Everywhere (ongoing). Provide examples for $n = 2, 3$ of as many results as we can.
        \item GRT.\@ Domains of defintion, i.e. $GRT: L^2 \rightarrow L^2$ and such.
        \item GRT.\@ Integration by parts formula for $GRT$ of $\partial f/\partial x_i$ (found dR/dp and thus dGR/dp)
        \item GRT.\@ Verify the conjecture for $n=3$
        \item GRT.\@ Prove the conjecture on GRT of Hermite polynomials.
        \item Try to derive GRT of monomials from Hermite polynomials.
        \item Mathematica. Implement Gaussian shape reconstruction method, try some example shapes. Clearly lay out the choices made: Series of expanding images, or compactified $\RR^n$? How to complexify points?
    \end{itemize}
\end{itemize}

\section{Shape Reconstruction}
The shape reconstruction method: In a 2005 paper, Annie Cuyt et.\ al.\ \cite{Cuyt05} proposed a method for shape reconstruction from moments, via Pade approximants to a multidimensional integral transform. Given a set of multivariate moments of some region $A$ in $\RR^n$, the method produces a pixel image approximating $A$. 

Suppose $A \subset \RR^n$, and let $f(x)$ be its indicator function. We may assume $f$ is measurable and has bounded support, and thus finite moments. The moment sequence gives Taylor coefficients in a neighborhood of zero for a certain holomorphic function with integral representation
\begin{align*}
    g(y) = \int_{\RR^n} \frac{f(x)}{1 + \langle x, y\rangle} ~dx \approx \sum_{\alpha \in \NN_0^n} \binom{|\alpha|}{\alpha} c_\alpha {(-y)}^\alpha.
\end{align*}
Thus the function $g(y)$ may be approximated to a certain degree of accuracy depending on the number of available moments.

At the same time $g$ can be approximated by a multivariate quadrature formula
\[
    \int_{\RR^n} \frac{f(x)}{1 + \langle x, y\rangle} ~dx
    = \sum_{i} \frac1{1+\langle x_i, y\rangle} f(x_i)
    = \sum_{i} w(x_i, y) f(x_i)
\]
Where the nodes $x_i$ lie, for example, on a cubic lattice. We can then sample a Pade approximant to $g$ at some sufficinetly large group of points $y = y_j$ forming a linear system of equations, from which we solve for $f(x_i)$. If all goes well we have approximations of $f(x_i)$ on a node lattice, which can be turned into a pixel image of $A$.

For now we will gloss over the numerical discussion of quadrature and linear systems, taking for granted that such methods are applicable in at least some simple cases. Computational tests (to be included in, say, section 10) further support the validity of the method. Our focus will be on demonstrating that one can approximate $g$ by rational Pad\'e approximants constructed from moments. To this end we show that, when restricted to one dimensional subspaces, $g$ is equivalent to the Stieltjes transform of the Radon transform of $f$ at a fixed projection angle. 
\[
    g(z\omega) = \int_{-\infty}^\infty \frac{R_f(\omega,p)}{1 + zp}dp
\]
where $\omega \in S^{n-1}, z\in \RR$. Thus we are able to leverage the more well understood properties of these univariate transforms. 

In particular, the ``projection moments'' of $R_f(\omega, p)$ in any fixed direction $\omega$, may be computed from the multivariate moments of $f$. Further, it can be shown that Pad\'e approximants to a moment sequence converge to the Stieltjes transform, and thus to $g$. Returning to shape reconstruction, it then follows that one can approximate $g$ at various sample points $y_j = z_j\omega_j$ via Pad\'e approximants on linear subspaces at a selection of angles $\omega$, as is needed to form our linear system.

Now let us briefly discuss our proposed modification to the shape reconstruction method, the theoretical complications and computational considerations. Essentially what we propose is to recreate the method with a Gaussian measure applied, thus allowing for convergence on a larger class of regions $A$. In particular, the proposed method would apply to unbounded regions, for which we cannot guarentee finite moments, convergence of the function $g(y)$ (let alone the explicit series expansion in terms of moments), or convergence of the Radon transform.

Here we will briefly discuss the potential challenges that this proposed method presents, which we address more carefully in sections 5 through 8. First and foremost instead of the standard moments (with respect to the Lebesgue measure), we are now given a Gaussian moment sequence — that is, moments with respect to the standard Gaussian measure on $\RR^n$. On the other end, the reconstructed region may be recovered as normal by simply inverting the Gaussian weight. 

From a theoretical standpoint, proving the validity of the proposed method presents a few challenges. Firstly, even a quick glance at the classical literature tells us that moment problems on bounded domains are substantially better behaved than their unbounded counterparts. Whereas all moment problems on a bounded interval are determinate, we now have the potential to run in to indeterminate problems. We show that we can guarentee determinacy in a Gaussian-weighted $L^2$ completion of the space of polynomials. This of course includes our primary use case: indicator functions. 

Secondly, we must address the effect of the proposed modifications on the Radon transform. We will define the Gaussian Radon transform, which unlike the Radon transform exists for unbounded regions $A$. 

The Stieltjes transform on unbounded domains is significantly more complicated than the original case. From the integral representation
\[
    g(y) = \int_{\RR^n} \frac{f(x)}{1 + \langle x, y\rangle} ~dx,
\]
one can start to see a problem: If $f(x)$ has unbounded support $g(y)$ has the potential to not converge for any $y \in \RR^n$. In particular $g(y)$ may not exist in a neighborhood of $0$, meaning our series expansion and thus the explicit conection to moments, may be broken. We need now to consider $g$ as a function on some non-real domain. Here we show Pade approximants can be computed from moments which still approximate $g$ on, for example, a half space. The convergence of these approximants is tied to the question determinacy of the moment problem, which we resolve as described above. Thus we simply take sample points avoiding the problematic real space and the method is validated.

